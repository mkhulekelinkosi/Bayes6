% Options for packages loaded elsewhere
\PassOptionsToPackage{unicode}{hyperref}
\PassOptionsToPackage{hyphens}{url}
%
\documentclass[
]{article}
\usepackage{amsmath,amssymb}
\usepackage{iftex}
\ifPDFTeX
  \usepackage[T1]{fontenc}
  \usepackage[utf8]{inputenc}
  \usepackage{textcomp} % provide euro and other symbols
\else % if luatex or xetex
  \usepackage{unicode-math} % this also loads fontspec
  \defaultfontfeatures{Scale=MatchLowercase}
  \defaultfontfeatures[\rmfamily]{Ligatures=TeX,Scale=1}
\fi
\usepackage{lmodern}
\ifPDFTeX\else
  % xetex/luatex font selection
\fi
% Use upquote if available, for straight quotes in verbatim environments
\IfFileExists{upquote.sty}{\usepackage{upquote}}{}
\IfFileExists{microtype.sty}{% use microtype if available
  \usepackage[]{microtype}
  \UseMicrotypeSet[protrusion]{basicmath} % disable protrusion for tt fonts
}{}
\makeatletter
\@ifundefined{KOMAClassName}{% if non-KOMA class
  \IfFileExists{parskip.sty}{%
    \usepackage{parskip}
  }{% else
    \setlength{\parindent}{0pt}
    \setlength{\parskip}{6pt plus 2pt minus 1pt}}
}{% if KOMA class
  \KOMAoptions{parskip=half}}
\makeatother
\usepackage{xcolor}
\usepackage[margin=1in]{geometry}
\usepackage{graphicx}
\makeatletter
\def\maxwidth{\ifdim\Gin@nat@width>\linewidth\linewidth\else\Gin@nat@width\fi}
\def\maxheight{\ifdim\Gin@nat@height>\textheight\textheight\else\Gin@nat@height\fi}
\makeatother
% Scale images if necessary, so that they will not overflow the page
% margins by default, and it is still possible to overwrite the defaults
% using explicit options in \includegraphics[width, height, ...]{}
\setkeys{Gin}{width=\maxwidth,height=\maxheight,keepaspectratio}
% Set default figure placement to htbp
\makeatletter
\def\fps@figure{htbp}
\makeatother
\setlength{\emergencystretch}{3em} % prevent overfull lines
\providecommand{\tightlist}{%
  \setlength{\itemsep}{0pt}\setlength{\parskip}{0pt}}
\setcounter{secnumdepth}{-\maxdimen} % remove section numbering
\ifLuaTeX
  \usepackage{selnolig}  % disable illegal ligatures
\fi
\IfFileExists{bookmark.sty}{\usepackage{bookmark}}{\usepackage{hyperref}}
\IfFileExists{xurl.sty}{\usepackage{xurl}}{} % add URL line breaks if available
\urlstyle{same}
\hypersetup{
  pdftitle={Untitled},
  pdfauthor={mkhulekeli Nkosi 2017159092},
  hidelinks,
  pdfcreator={LaTeX via pandoc}}

\title{Untitled}
\author{mkhulekeli Nkosi 2017159092}
\date{2025-04-28}

\begin{document}
\maketitle

The Laplace transform is a powerful tool in mathematics, particularly
useful for solving differential equations and analyzing systems. To find
the distribution of a random variable using the Laplace transform, you
typically follow these steps:

\begin{enumerate}
\def\labelenumi{\arabic{enumi}.}
\tightlist
\item
  Define the Random Variable
\end{enumerate}

Let ( X ) be a continuous random variable with probability density
function (PDF) ( f\_X(x) ).

\begin{enumerate}
\def\labelenumi{\arabic{enumi}.}
\setcounter{enumi}{1}
\tightlist
\item
  Laplace Transform of the PDF
\end{enumerate}

The Laplace transform of the PDF ( f\_X(x) ) is given by:
\[ \mathcal{L}{f_X(x)}(s) = \int_{0}^{\infty} e^{-sx} f_X(x) , dx \]

\begin{enumerate}
\def\labelenumi{\arabic{enumi}.}
\setcounter{enumi}{2}
\tightlist
\item
  Find the Laplace Transform
\end{enumerate}

Compute the Laplace transform of the PDF. For example, if ( f\_X(x) ) is
an exponential distribution with parameter ( \lambda ), then:
\[ f_X(x) = \lambda e^{-\lambda x} \quad \text{for} \quad x \geq 0 \]
The Laplace transform is:
\[ \mathcal{L}{f_X(x)}(s) = \int_{0}^{\infty} e^{-sx} \lambda e^{-\lambda x} , dx = \frac{\lambda}{s + \lambda} \]

\begin{enumerate}
\def\labelenumi{\arabic{enumi}.}
\setcounter{enumi}{3}
\tightlist
\item
  Inverse Laplace Transform
\end{enumerate}

To find the original distribution from its Laplace transform, you need
to perform the inverse Laplace transform. The inverse Laplace transform
is given by: \[ f_X(x) = \mathcal{L}^{-1}{F(s)}(x) \] where ( F(s) ) is
the Laplace transform of ( f\_X(x) ).

Example

Let's consider an example where ( X ) follows an exponential
distribution with parameter ( \lambda ).

PDF: ( f\_X(x) = \lambda e\^{}\{-\lambda x\} ) Laplace Transform:
\[ \mathcal{L}{f_X(x)}(s) = \int_{0}^{\infty} e^{-sx} \lambda e^{-\lambda x} , dx = \frac{\lambda}{s + \lambda} \]
Inverse Laplace Transform:
\[ f_X(x) = \mathcal{L}^{-1}\left{\frac{\lambda}{s + \lambda}\right}(x) = \lambda e^{-\lambda x} \]

Thus, we have recovered the original exponential distribution.

Conclusion

Using the Laplace transform to find the distribution involves
transforming the PDF, computing the Laplace transform, and then applying
the inverse Laplace transform to retrieve the original distribution.
This method is particularly useful for solving complex differential
equations and analyzing the behavior of systems in engineering and
physics. If you have a specific distribution or problem in mind, feel
free to share, and I can provide more tailored guidance!

\end{document}
